\documentclass[10pt]{article}
\usepackage{geometry}                
\geometry{a4paper}
\usepackage{graphicx}
\usepackage{amssymb}
\usepackage{amsmath}
\usepackage{paralist}
\usepackage[parfill]{parskip}

\title{TMA4215 Numerical Mathematics: Project 2 \\ Replace this line with your own title} 
\author{Student nr. 6XXXXX and 7XXXXX} % Here you insert the student number of the members of the group.
\date{\today}

                                          
\begin{document}
\maketitle
\begin{abstract}

Bla bla Her kjem Abstract, men eg berre skriv litt for aa setje av plass til det. \\
Jepsi pepsi \\
Bezier er bra \\
Hurra \\

\end{abstract}

\section{Introduction} 

Bla bla Her kjem Introduction, men eg berre skriv litt for aa setje av plass til det. \\
Jepsi pepsi \\
Bezier er bra \\
Hurra \\
Tralalala \\
Eg er litt gal. \\

\section{Theory}

\subsection*{Bezier curves and Bernstein polynomials}

We have used Bezier curves to approximate a given curve $\gamma$. These are parametrized curves consisting of sums of Bernstein polynomials. A Bernstein polynomial is a polynomial of the form
\begin{align}
B_{i}^n(t) = \binom{n}{i}t^i(t-1)^{(n-i)},\quad i = 0, 1, ..., 0,\quad t \; \in \; [0,1],
\end{align}
and a bezier curve of degree n is given by
\begin{align}
\mathbf{S}(t) = \sum_{i=0}^{n} \mathbf{P_i} B_{i}^n(t), \quad 0 \leq t \leq 1,
\end{align}
where $\mathbf{P_i}$ are so called control. These points are chosen such that $\mathbf{P_0}$ and $\mathbf{P_n}$ are interpolation points between the Bezier curve and $\gamma$. In this report we are using cubic Bezier curves for solving the curve fitting problem, which means that each Bezier curves is made out of 4 points. $\mathbf{P_0}$ and $\mathbf{P_3}$ are endpoints, and $\mathbf{P_1}$ and $\mathbf{P_2}$ are points giving the curvature.

\subsection*{The least mean square problem}

For the best fit, $\mathbf{P_1}$ and $\mathbf{P_2}$ have to be chosen such that the distance between $\gamma$ and our bezier curve is the small. That is, we needed to minimize the the following expression:

\begin{equation}
E = \frac{1}{m} \sum_{i=1}^{m} \| \mathbf{x}_i - \mathbf{S}(t_i)\|^2_2,
\end{equation}

for parametrization values $t_i \in [ 0,1 ]$ and points $\mathbf{x_i}$ on $\gamma$. To do this we use a property of the Bezier curves: The two straight lines between $\mathbf{P_0}$ and $\mathbf{P_1}$, and $\mathbf{P_2}$ and $\mathbf{P_3}$ are tangents to the curve at $\mathbf{P_0}$ and $\mathbf{P_3}$ respectively:

\begin{align}
\mathbf{P_1} = \mathbf{P_0} + \alpha_1 \mathbf{v_0} \\
\mathbf{P_2} = \mathbf{P_3} + \alpha_2 \mathbf{v_3}
\end{align}

Since $\mathbf{P_0}$ and $\mathbf{P_3}$ are known from the points on the curve $\gamma$, this problem is reduced to finding $\alpha_1$ and $\alpha_2$ such that $E$ is small. We differentiate $E$ with respect to $\alpha_1$ and $\alpha_2$. To ease the reader we omit the whole process, and only state the resulting system of equations for $\alpha_1$ and $\alpha_2$:

\begin{align}
\alpha_1 \sum_{i = 1}^m B_1(t_i)^2 (v_{0x}^2 + v_{0y}^2) + \alpha_2 \sum_{i = 1}^m B_1(t_i)B_2(t_i)(v_{0x}v_{3x} + v_{0y}v_{3y}) \\
= - \sum_{i = 0}^m B_1(t_i)(v_{0x}A_x + v_{0y}A_y) \\
\alpha_1 \sum_{i = 1}^m B_1(t_i)B_2(t_i)(v_{0x}v_{3x} + v_{0y}v_{3y}) + \alpha_2 \sum_{i = 1}^m B_2(t_i)^2 (v_{3x}^2 + v_{3y}^2) \\
= - \sum_{i = 0}^m B_1(t_i)(v_{3x}A_x + v_{3y}A_y)
\end{align}

\begin{align}
\alpha_1 \sum_{i = 1}^m B_1(t_i)^2 |\mathbf{v_0}|^2 + \alpha_2 \sum_{i = 1}^m B_1(t_i)B_2(t_i)\mathbf{v_0} \cdot \mathbf{v_3} 
= - \sum_{i = 0}^m B_1(t_i) \mathbf{v_0} \cdot \mathbf{A_i}, \\
\alpha_1 \sum_{i = 1}^m B_1(t_i)B_2(t_i)\mathbf{v_0} \cdot \mathbf{v_3} + \alpha_2 \sum_{i = 1}^m B_2(t_i)^2 |\mathbf{v_3}|^2 
= - \sum_{i = 0}^m B_1(t_i)\mathbf{v_3} \cdot \mathbf{A_i},
\end{align}

where $\mathbf{A_i} = \mathbf{P_0}(B_0(t_i) + B_1(t_i) + \mathbf{P_3}(B_3(t_i) + B_2(t_i)) - \mathbf{x_i}$.




Mean square problem-----------------

To measure the distance between the original curve, $\gamma$, and the Bezier curves, we used the mean square distance 

\begin{equation}
E = \frac{1}{n} \sum_{i=1}^{n} \|\mathbf{x}_i - \mathbf{S}(t_i)\|^2_2,
\end{equation}

where $n$ is the number of points $\mathbf{x}_i$ on the curve. 

parametrization of t
Newton Raphson

Choosing the initial $t$-values can be done in many ways. The easiest is probably to generate values ranging from 0 to 1 with equal distance, using the known command "linspace" in MATLAB. But when measuring the distance between $\mathbf{x}_i$ and the respective value $\mathbf{S}(t_i)$, we cannot be sure this is the minimal distance between the Bezier curve and $\gamma$. If we can manage to displace the $t$-values such that the distance between two respective points are minimal, the measured error is much more precise. This can be done using Newton-Raphson iteration. But for this iteration to converge, we need sufficiently good starting values. We use initial $t$-values given by the distance between the data points from the curve $\gamma$. This method is merely based on \cite{Plass:1983}. Let

\begin{equation}
s_k = \sum_{i=1}^{k-1} \sqrt{(x_{i+1}-x_i)^2 + (y_{i+1}-y_i)^2}
\end{equation}

be the length of the polygonal segment from 1 to $k$, $k$ = 1, 2, ..., $n$. The initial parametric value is then $t_k = s_k/s_n$.

Newton-Raphson is a fixed point iteration method. Our goal is to find $t$-values such that the square distance between $\gamma$ and the Bezier curve is at its smallest. By  differentiating the square distance and equate it to zero, we find its minimum. Let $f(t) = \frac{d}{dt} [(X(t)-x)^2 + (Y(t)-y)^2 ]$ be the differentiated square distance. If Newton-Raphson does converge for the initial $t$-values, we will be able to find the root of $f(t) = 0$ with the equation

\begin{equation}
t = t - \frac{f(t)}{f'(t)}.
\end{equation}

This method converges quickly, so few steps are provided for a $t$-value close enough to the root. It is important that the values are still in range 0 to 1. With the new $t$-values, we can check if the error is small enough to make a Bezier curve. If not, it is a good idea to split the interval in two, and do new approximations for the $t$-values.








Divide curve in two, recursive, largest distance

Tangent vectors
Corners, control points
Number of Bezier curves (Numerical section?
Figures are good!


\section{Numerical experiments}

In this section we will blahblahblah. 



\section{Conclusion}
\cite{Plass:1983}

\bibliographystyle{plain}
\bibliography{bezierbib}   % Insert you own BibTeX file 
\end{document}  
