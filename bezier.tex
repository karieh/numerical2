\documentclass[10pt]{article}
\usepackage{geometry}                
\geometry{a4paper}
\usepackage{graphicx}
\usepackage{amssymb}
\usepackage{amsmath}
\usepackage{paralist}
\usepackage[parfill]{parskip}

\title{TMA4215 Numerical Mathematics: Project 2 \\ Replace this line with your own title} 
\author{Student nr. 6XXXXX and 7XXXXX} % Here you insert the student number of the members of the group.
\date{\today}

                                          
\begin{document}
\maketitle
Hello
\begin{abstract}

\end{abstract}

\section{Introduction} 

\section{Theory}

\subsection*{Bezier curves and Bernstein polynomials}

Bezier curves consist of a sum of Bernstein polynomials. A Bernstein polynomial is a polynomial of the form
\begin{align}
B_{i}^n(t) = \binom{n}{i}t^i(t-1)^{(n-i)},\quad i = 0, 1, ..., 0,\quad t \; \in \; [0,1].
\end{align}



To make a parametrized curve for fitting a shape we have used cubic Bezier curves. 

Bezier curves

Mean square problem-----------------

To measure the distance between the original curve, $\gamma$, and the Bezier curves, we used the mean square distance 

\begin{equation}
E = \frac{1}{n} \sum_{i=1}^{n} \|\mathbf{x}_i - \mathbf{S}(t_i)\|^2_2,
\end{equation}

where $n$ is the number of points $\mathbf{x}_i$ on the curve. 

parametrization of t
Newton Raphson

Choosing the initial $t$-values can be done in many ways. The easiest is probably to generate values ranging from 0 to 1 with equal distance, using the known command "linspace" in MATLAB. But when measuring the distance between $\mathbf{x}_i$ and the respective value $\mathbf{S}(t_i)$, we cannot be sure this is the minimal distance between the Bezier curve and $\gamma$. If we can manage to displace the $t$-values so that the distance between two respective points are minimal, the measured error is much more precise. This can be done using Newton-Raphson iteration. But for this iteration to converge, we need sufficiently good starting values. We use initial $t$-values given by the distance between the data points from the curve $\gamma$. This method is partly based on \cite{Plass:1983}. Let

\begin{equation}
s_k = \sum_{i=1}^{k-1} \sqrt{(x_{i+1}-x_i)^2 + (y_{i+1}-y_i)^2}
\end{equation}

be the length of the polygonal segment from 1 to $k$, $k$ = 1, 2, ..., $n$. The initial parametric value is then $t_k = s_k/s_n$.

Newton-Raphson is a fixed point iteration 

For our parametrization values $t$, we use  To make our parametrization curve closer to $\gamma$, we used 






Divide curve in two, recursive, largest distance

Tangent vectors
Corners, control points
Number of Bezier curves (Numerical section?
Figures are good!


\section{Numerical experiments}

In this section we will blahblahblah. 



\section{Conclusion}
\cite{Plass:1983}

\bibliographystyle{plain}
\bibliography{bezierbib}   % Insert you own BibTeX file 
\end{document}  
