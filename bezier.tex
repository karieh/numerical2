\documentclass[10pt]{article}
\usepackage{geometry}                
\geometry{a4paper}
\usepackage{graphicx}
\usepackage{amssymb}
\usepackage{amsmath}
\usepackage{paralist}
\usepackage[parfill]{parskip}

\title{TMA4215 Numerical Mathematics: Project 2 \\ Replace this line with your own title} 
\author{Student nr. 6XXXXX and 7XXXXX} % Here you insert the student number of the members of the group.
\date{\today}

                                          
\begin{document}
\maketitle
Hello
\begin{abstract}

\end{abstract}

\section{Introduction} 

\section{Theory}

\subsection*{Bezier curves and Bernstein polynomials}

Bezier curves consist of a sum of Bernstein polynomials. A Bernstein polynomial is a polynomial of the form
\begin{align}
B_{i}^n(t) = \binom{n}{i}t^i(t-1)^{(n-i)},\quad i = 0, 1, ..., 0,\quad t \; \in \; [0,1].
\end{align}



To make a parametrized curve for fitting a shape we have used cubic Bezier curves. 

Bezier curves
Mean square problem
parametrization of t
Newton Raphson
Divide curve in two, recursive, largest distance

Tangent vectors
Corners, control points
Number of Bezier curves (Numerical section?
Figures are good!


\section{Numerical experiments}

In this section we will blahblahblah. 



\section{Conclusion}
\cite{Plass:1983}

\bibliographystyle{plain}
\bibliography{bezierbib}   % Insert you own BibTeX file 
\end{document}  
